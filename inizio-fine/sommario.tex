% !TEX encoding = UTF-8
% !TEX TS-program = pdflatex
% !TEX root = ../tesi.tex

%**************************************************************
% Sommario
%**************************************************************
\cleardoublepage
\phantomsection
\pdfbookmark{Sommario}{Sommario}
\begingroup
\let\clearpage\relax
\let\cleardoublepage\relax
\let\cleardoublepage\relax

\chapter*{Sommario}

Il presente documento descrive il lavoro svolto durante il periodo di stage, dalla durata di circa trecentoquaranta ore, del laureando Carlo Fontolan presso l'azienda Siav S.p.A.
Lo studente vuole descrivere con riflessioni critiche e oggettive quanto appreso e maturato dal punto di vista professionale e delle competenze in tutto l'arco di durata dello stage.
In primo luogo era richiesto lo sviluppo di una libreria di filtraggio in abito process mining, che verrà poi inserità all'interno di una architettura a microservizi.
In secondo luogo, è stata richiesta le riscrittura di alcune sezioni di un loro vecchio prodotto, in cui la libreria sopradescrittà farà da protagonista, per quanto riguarda la sezione di filtraggio. Le caratteristiche richieste dall’azienda proponente erano che il prodotto fosse idoneo a girare su cloud e che fosse costituito da un’architettura a microservizi e sviluppato con framework all'avanguardia.
Siccome in corso d'opera sono sorte alcune criticità in merito allo sviluppo della libreria, in accordo con il tutor aziendale, abbiamo deciso di rinegoziare alcuni requisiti andando ad eliminare la sezione riguardante il  \textit{backend}, facendo spazio ad alcuni stub in modo da poter osservare il comportamento dell'interfaccia, soddisfando così gli obiettivi sia a livello aziendale che personale. 


%\vfill
%
%\selectlanguage{english}
%\pdfbookmark{Abstract}{Abstract}
%\chapter*{Abstract}
%
%\selectlanguage{italian}

\endgroup			

\vfill

