
%**************************************************************
% Glossario
%**************************************************************
%\renewcommand{\glossaryname}{Glossario}

\newglossaryentry{apig}
{
    name=\glslink{api}{API},
    text=Application Program Interface,
    sort=api,
    description={in informatica con il termine \emph{Application Programming Interface API} (ing. interfaccia di programmazione di un'applicazione) si indica ogni insieme di procedure disponibili al programmatore, di solito raggruppate a formare un set di strumenti specifici per l'espletamento di un determinato compito all'interno di un certo programma. La finalità è ottenere un'astrazione, di solito tra l'hardware e il programmatore o tra software a basso e quello ad alto livello semplificando così il lavoro di programmazione}
}



\newglossaryentry{helpdesk}
{
    name=\glslink{helpdesk}{Helpdesk},
    text=helpdesk,
    sort=helpdesk,
    description={Servizio clienti telematico}
}

\newglossaryentry{tfs}
{
    name=\glslink{tfs}{TFS},
    text=TFS,
    sort=Tfs,
    description={Team foundation server: prodotto Microsoft per la gestione del ciclo di vita dei progetti software}
}

\newglossaryentry{tfvc}
{
    name=\glslink{tfvc}{TFVC},
    text=TFVC,
    sort=Tfvc,
    description={Team foundation version control: strumento di gestione del versionamento del codice sorgente sviluppato per Team foundation server}
}
\newglossaryentry{deploy}
{
    name=\glslink{deploy}{Deploy},
    text=deploy,
    sort=deploy,
    description={ consegna o rilascio al cliente, con relativa installazione e messa in funzione o esercizio, di una applicazione o di un sistema software tipicamente all'interno di un sistema informatico aziendale}
}

\newglossaryentry{Angular}
{
    name=\glslink{Angular}{Angular},
    text=Angular,
    sort=Angular,
    description={ Angular 2+ è una piattaforma open source per lo sviluppo di applicazioni web con licenza MIT, evoluzione di AngularJS. Sviluppato principalmente da Google, la sua prima release è avvenuta il 14 settembre 2016}
}

\newglossaryentry{Websocket}
{
    name=\glslink{Websocket}{Websocket},
    text=Websocket,
    sort=Websocket,
    description={tecnologia web che fornisce canali di comunicazione full-duplex attraverso una singola connessione TCP}
}
\newglossaryentry{Cloud}
{
    name=\glslink{Cloud}{Cloud},
    text=Cloud,
    sort=Cloud,
    description={paradigma di erogazione di servizi offerti on demand da un fornitore ad un cliente finale attraverso la rete Internet}
}

\newglossaryentry{BigData}
{
    name=\glslink{BigData}{BigData},
    text=BigData,
    sort=BigData,
    description={raccolta estesa di dati in termini di volume, velocità e varietà da richiedere tecnologie e metodi analitici specifici per l'estrazione di valore o conoscenza}
}

\newglossaryentry{Framework}
{
    name=\glslink{Framework}{Framework},
    text=Framework,
    sort=Framework,
    description={Architettura logica di supporto su cui un software può essere progettato e realizzato}
}

\newglossaryentry{Broker}
{
    name=\glslink{Broker}{Broker},
    text=Broker,
    sort=Broker,
    description={Sistema che distribuisce vari aspetti del software su nodi differenti tramite l'utilizzo di oggetti remoti}
}

\newglossaryentry{Stage-IT}
{
    name=\glslink{Stage-IT}{Stage-IT},
    text=Stage-IT,
    sort=Stage-IT,
    description={Evento organizzate presso Padova Fiere in collaborazione con varie aziende locali che cercano di far approcciare gli studenti al mondo lavorativo tramite attività di stage}
}
\newglossaryentry{Javadoc}
{
    name=\glslink{Javadoc}{Javadoc},
    text=Javadoc,
    sort=Javadoc,
    description={Javadoc è un applicativo incluso all'interno del Java Development Kit della Sun Microsystems, utilizzato per la generazione automatica della documentazione del codice sorgente scritto in linguaggio Java}
}
\newglossaryentry{stub}
{
    name=\glslink{stub}{Stub},
    text=stub,
    sort=stub,
    description={è una porzione di codice utilizzata in sostituzione di altre funzionalità software in quanto può simulare il comportamento di codice esistente o l'interfaccia COM, e temporaneo sostituto di codice ancora da sviluppare}
}
\newglossaryentry{callback}
{
    name=\glslink{callback}{Callback},
    text=callback,
    sort=callback,
    description={funzione o blocco di codice che viene passato come parametro ad un'altra funzione}
}
\newglossaryentry{Disco}
{
    name=\glslink{Disco}{Disco},
    text=Disco,
    sort=Disco,
    description={Tool di process mining sviluppato da Fluxicon}
}

\newglossaryentry{JSON}
{
    name=\glslink{JSON}{JSON},
    text=JSON,
    sort=JSON,
    description={JavaScript Object Notation. Formato adatto all'interscambio di applicazione client/server}
}

\newglossaryentry{unit test}
{
    name=\glslink{unit test}{Unit test},
    text=unit test,
    sort=unit test,
    description={l'attività di testing di singole unità software.}
}
\newglossaryentry{mockup}
{
    name=\glslink{mockup}{Mockup},
    text=mockup,
    sort=mockup,
    description={realizzazione a scopo illustrativo di un oggetto o sistema senza le complete funzioni originali}
}
\newglossaryentry{open-closed}
{
    name=\glslink{open-closed}{Open-closed},
    text=open-closed,
    sort=open-closed,
    description={principio cardine del S.O.L.I.D \textit{principle} attraverso il quale un sistema deve essere aperto alle estensioni ma chiuso alle modifiche.}
}

\newglossaryentry{OOP}
{
    name=\glslink{OOP}{OOP},
    text=OOP,
    sort=OOP,
    description={Object Orientation Principle: }
}