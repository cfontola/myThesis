% !TEX encoding = UTF-8
% !TEX TS-program = pdflatex
% !TEX root = ../tesi.tex

%**************************************************************
\chapter{Valutazione retrospettiva}
\label{cap:valutazione-retrospettiva}
%**************************************************************


\section{Soddisfacimento obiettivi}
Gli obiettivi concordati con il tutor in seguito alle negoziazione dei requisiti sono stati portati a termine in maniera completa, ciò ho permesso di arrivare ad un buon grado di soddisfacimento sia personale, sia da parte del tutor.\\
Purtroppo alcune aspettative che mi ero prefissato all'inizio dello stage non state rispecchiate a pieno. Questo è stato causato da alcune problematiche che sono sorte in corso d'opera, causando un rallentamento della attività pianificate e la conseguente rinegoziazione dei requisiti. Ciò nonostante mi ha permesso di arrivare alla fine delle attività con un notevole aumento del mio bagaglio conoscitivo, dovuto anche ad una maggior consapevolezza di saper produrre prodotti di qualità anche di fronte a problematiche che spesso accadono all'interno di analoghi contesti lavorativi. Dal punto di vista degli obiettivi aziendali, le attività svolte hanno portato ad uno stato di avanzamento sostanziale del prodotto finale, includendo tutti i principali aspetti che riguardano la pulizia del log degli eventi, andando quindi a soddisfare tutti gli obiettivi obbligatori che sono stati concordati con il tutor, indicati all'interno del piano di lavoro. Per quanto riguarda gli obiettivi desiderabili non sono riuscito a soddisfarli in maniera completa; questo perchè ho dovuto affrontare numerevoli discussioni in merito ad aspetti rigurdati la progettazione della libreria. Ciò ha comportatato un cospicuo investimento di risorse che non erano state preventivate inizialmente, andando quindi a dover trascurare alcuni aspetti marginali al fine del soddisfacimento degli obiettivi.
\section{Conoscenze acquisite}
\subsection{Dominio del \textit{process mining}}
Prima dello stage, il dominio del \textit{process mining} era un argomento completamente sconosciuto, ma grazie a questi due mesi trascorsi in \textit{Siav} ho potuto comprendere le sue principali meccaniche, applicate nel concreto ad un contesto aziendale ben organizzato. Inizialmente, dopo il periodo di formazione, il concetto era molto generico e non di facile comprensione, ma tramite esempi pratici applicati al contesto aziendale ho avuto un notevole incremento delle mie conoscenze sia dal punto di vista teorico che applicativo.
\subsection{Angular}
Prima di interfacciarmi allo sviluppo di \textit{webapp}, la mia visione di sviluppo web non si era allontanata di molto rispetto a quanto studiato durante il corso di tecnologie web svolto durante il terzo anno. Nel periodo trascorso all'interno dell'azienda ho avuto l'occasione di apprendere le principali dinamiche di sviluppo di una \textit{webapp}, non soffermandomi quindi solo al concetto di \textit{html} e \textit{css}, ma integrando tali concetti con un livello di astrazione più alto.
\subsection{\textit{Websocket}}
Tale concetto mi era completamente sconosciuto prima dello stage, è stato possibile però interfacciarmi ad esso, anche se parizalmente, cercando di capirne il campo applicativo e gli eventuali vantaggi che può portare una scelta di tale protocollo di trasmissione dati. Ciò è stato possibile tramite lo sviluppo di uno \textit{stub} che mi ha fatto intendere in maniera generica come tale protocollo vada ad interfacciarsi con le varie tipologie di richieste.
\subsection{Ambiente lavorativo}
Nel tempo trascorso presso \textit{Siav} ho potuto osservare e partecipare ad alcune dinamiche presenti all'interno del loro contesto aziendale. Tali principi e meccanismi erano sconosciuti per me fin ad ora, essendo sempre stato all'interno di una realtà universitaria che differiva in maniera consistente rispetto alla realtà in cui sono stato inserito. Così facendo ho potuto constatare, anche se in forma ancora molto generica, cosa comporti essere all'interno di una realtà lavorativa ben organizzata e strutturata come quella in cui sono stato inserito, includendo tutte le difficoltà, le aspettative, i processi e le tempistiche che la compongono.
\section{Relazione tra lavoro e università}
Una buona parte delle tematiche che sono affrontate all'interno della realtà aziendale da me osservate non vengono trattate all'interno del corso di studi che ho intrapreso. Ciò nonostante non ha influito sulla comprensione delle principali metodologie trattate: mi ha portato ad un miglioramento sostanziale del mio grado di formazione, consentendomi un inserimento più concreto all'interno del contesto lavorativo.
Il percorso di studi da me affrontato è stato di notevole aiuto per poter apprendere alcuni aspetti in maniera sistematica e trasparente. Un esempio riguarda i principali strumenti utilizzati, alcuni erano gia stati usati in maniera concreta durante la mia esperienza universitaria, specialmente durante il corso di ingegneria del software; altri sono stati utilizzati per la prima volta proprio in questo contesto. Ciò comunque non ha causato particolari complicazioni, dovute all'esprienza maturata in merito allo svolgimento di vari progetti durante il corso degli studi. Un'ulteriore tematica che ho trovato di grande aiuto durante lo svolgimento dello stage sono stati i principi della programmazione ad oggetti (\gls{OOP}), tramite i quali ho potuto sviluppare una libreria solida che aderisse ai suoi principali fondamenti. Ho trovato invece una maggior difficoltà nel comprendere alcune dinamiche specifiche all'interno di un ambito lavorativo di questo genere, ciò è dovuto alla mia totale inesperienza a tal merito. Nel complesso sono soddisfatto di quanto appreso durante questa esperienza formativa, avendo aumentato in maniera cospicua il mio bagaglio di conoscenze, integrandolo con quanto già in mio possesso. 
