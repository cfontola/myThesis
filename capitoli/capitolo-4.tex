% !TEX encoding = UTF-8
% !TEX TS-program = pdflatex
% !TEX root = ../tesi.tex

%**************************************************************
\chapter{Valutazione retrospettiva}
\label{cap:analisi-requisiti}
%**************************************************************


\section{Soddifacimento obiettivi}
Gli obiettivi prefissati con il tutor in seguito alle negoziazione dei requisiti sono stati portati a termine in maniera completa, ciò ho permesso di arrivare ad un buon grado di soddisfacimento sia personale, sia dalla parte del tutor.\\
Purtroppo alcune aspettative che mi ero prefissato all'inizio dello stage non state rispecchiate a pieno. Questo è stato causato da alcune problematiche che sono sorte in corso d'opera con il consenguente rallentamento della attività pianificate e la conseguente rinegoziazione dei requisiti. Ciò nonostante mi ha permesso di arrivare alla fine delle attività con un notevole aumento della mia consapevolezza di saper produrre prodotti di qualità anche di fronte a problematiche che spesso accadono all'interno di questi contesti lavorativi. Dal punto di vista degli obiettivi aziendali, le mie attività hanno portato ad una maturazione di quello che poi sarà il prodotto finale, includendo tutti i principali aspetti che riguardano la pulizia del log degli eventi, andando quindi a soddisfare tutti gli obiettivi obbligato ri che sono stati indicati all'interno del piano di lavoro. Per quanto riguarda gli obiettivi desiderabili non sono riuscito a soddisfarli in maniera completa a causa di altre esigenze che sono state messe in primo piano. 
\section{Conoscenze acquisite e difficoltà riscontrate}
L'attività in sè mi ha permesso di poter comprendere le principali dinamiche che può presentare una tecnica di analisi di processi, capendone le sue dinamiche e la sua utilità. Inizialmente, dopo la fase iniziale di formazione, il concetto era molto generico e non di facile comprensione, ma tramite esempi pratici applicati al contesto aziendale ho avuto un notevole incremento delle mie conoscenze sia dal punto di visto teorico che applicativo. Per quanto riguarda il lato Angular, ho appreso le principali tecniche di analisi, progettazione e codifica di \textit{webapp} integrata con i relativi servizi \textit{websocket}. Quest'ultimi sono stati compresi in modo lineare, andando a 

\section{L'esperienza in relazione all'ambiente universitario}
In generale tale esperienza è stata positiva per me, permettomi di entrare direttamente a contatto con il settore dell'informatica. 