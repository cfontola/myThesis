% !TEX encoding = UTF-8
% !TEX TS-program = pdflatex
% !TEX root = ../tesi.tex

%**************************************************************
\chapter{Introduzione}

\section{L'azienda}

Siav è una delle più importanti realtà italiane di sviluppo software e di servizi informatici
specializzata nella dematerializzazione e nella gestione documentale e nei processi
digitali. Presenta diverse filiali nel suolo italiano e cooperano tra loro per un fine comune.
All'interno dello sede di sviluppo di Rubano erano presenti diversi reparti, ognuno con una mansione ed un scopo ben preciso dedicati ai vari software che fornisce l'azienda ai propri clienti.  
Essendo stato inserito sin da subito all'interno del team di ricerca e sviluppo (di cui il mio tutor ne è team leader) ho potuto vedere nel concreto la costante crescita ed espansione dell'azienda, che va ricercando costantemente nuove idee e strategie che spesso vengono proposte tramite attività di stage.

\begin{figure}[!h] 
	\centering 
	\includegraphics[width=0.3\columnwidth]{logo-siav} 
	\caption{Logo di Siav Spa}
\end{figure}

%**************************************************************
\section{L'idea}
Il process mining è una disciplina di Business Intelligence per l’analisi ed il monitoraggio di
processi aziendali. In particolare, le tecniche di process mining utilizzano analisi statistiche per estrarre informazioni utili dai dati di log di un’azienda.
La pulizia dei dati in ingresso è un’operazione fondamentale di qualsiasi workflow di analisi.
Questo è valido anche nel caso del Process Mining, dove ne abbiamo bisogno per due motivi
principali: eliminare eventuali dati superflui dagli eventi raccolti e restringere la nostra analisi a sottocasi ben precisi, consentendo di portare il focus sulle prospettive desiderate.
Da tempo l'azienda aveva sviluppato un applicativo in grado di effettuare molteplici oprazioni su log degli eventi, anche se mal scritto e poco estendibile. L'idea che mi è stata proposta è stata quella di realizzare una libreria, che sarà successivamente inserita in un'architettura a microservizi, per poter effettuare operazioni di filtraggio, puntando soprattutto sull'estendibilità e mantenibiltà del codice in modo da poter rimpiazzare in futuro l'applicativo esistente. L'idea in sè è stata molto apprezzata sia per quanto riguardo l'argomento del process mining, sia per le nuove tecnologie che ho utilizzato. 

%**************************************************************
\section{Organizzazione del testo}

\begin{description}
    \item[{\hyperref[cap:processi-metodologie]{Il secondo capitolo}}] descrive le tecniche e le metodologie adottate per lo sviluppo del progetto
    
    \item[{\hyperref[cap:descrizione-stage]{Il terzo capitolo}}] approfondisce ...
    
    \item[{\hyperref[cap:analisi-requisiti]{Il quarto capitolo}}] approfondisce ...
    
    \item[{\hyperref[cap:progettazione-codifica]{Il quinto capitolo}}] approfondisce ...
    
    \item[{\hyperref[cap:verifica-validazione]{Il sesto capitolo}}] approfondisce ...
    
    \item[{\hyperref[cap:conclusioni]{Nel settimo capitolo}}] descrive ...
\end{description}

Riguardo la stesura del testo, relativamente al documento sono state adottate le seguenti convenzioni tipografiche:
\begin{itemize}
	\item gli acronimi, le abbreviazioni e i termini ambigui o di uso non comune menzionati vengono definiti nel glossario, situato alla fine del presente documento;
	\item per la prima occorrenza dei termini riportati nel glossario viene utilizzata la seguente nomenclatura: \emph{parola}\glsfirstoccur;
	\item i termini in lingua straniera o facenti parti del gergo tecnico sono evidenziati con il carattere \emph{corsivo}.
\end{itemize}