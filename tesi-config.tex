%**************************************************************
% file contenente le impostazioni della tesi
%**************************************************************

%**************************************************************
% Frontespizio
%**************************************************************

% Autore
\newcommand{\myName}{Fontolan Carlo}                                    
\newcommand{\myTitle}{Realizzazione a microservizi di una applicazione di process mining per il filtraggio degli eventi}

% Tipo di tesi                   
\newcommand{\myDegree}{Tesi di laurea triennale}

% Università             
\newcommand{\myUni}{Università degli Studi di Padova}

% Facoltà       
\newcommand{\myFaculty}{Corso di Laurea in Informatica}

% Dipartimento
\newcommand{\myDepartment}{Dipartimento di Matematica "Tullio Levi-Civita"}

% Titolo del relatore
\newcommand{\profTitle}{Prof.}

% Relatore
\newcommand{\myProf}{Tullio Vardanega}

% Luogo
\newcommand{\myLocation}{Padova}

% Anno accademico
\newcommand{\myAA}{2018-2019}

% Data discussione
\newcommand{\myTime}{Dicembre 2019}


%**************************************************************
% Impostazioni di impaginazione
% see: http://wwwcdf.pd.infn.it/AppuntiLinux/a2547.htm
%**************************************************************

\setlength{\parindent}{14pt}   % larghezza rientro della prima riga
\setlength{\parskip}{0pt}   % distanza tra i paragrafi


%**************************************************************
% Impostazioni di biblatex
%**************************************************************
\bibliography{bibliografia} % database di biblatex 

\defbibheading{bibliography} {
    \cleardoublepage
    \phantomsection 
    \addcontentsline{toc}{chapter}{\bibname}
    \chapter*{\bibname\markboth{\bibname}{\bibname}}
}

\setlength\bibitemsep{1.5\itemsep} % spazio tra entry

\DeclareBibliographyCategory{opere}
\DeclareBibliographyCategory{web}

\addtocategory{opere}{womak:lean-thinking}
\addtocategory{web}{site:agile-manifesto}

\defbibheading{opere}{\section*{Riferimenti bibliografici}}
\defbibheading{web}{\section*{Siti Web consultati}}


%**************************************************************
% Impostazioni di caption
%**************************************************************
\captionsetup{
    tableposition=top,
    figureposition=bottom,
    font=small,
    format=hang,
    labelfont=bf
}

%**************************************************************
% Impostazioni di glossaries
%**************************************************************

%**************************************************************
% Glossario
%**************************************************************
%\renewcommand{\glossaryname}{Glossario}

\newglossaryentry{apig}
{
    name=\glslink{api}{API},
    text=Application Program Interface,
    sort=api,
    description={in informatica con il termine \emph{Application Programming Interface API} (ing. interfaccia di programmazione di un'applicazione) si indica ogni insieme di procedure disponibili al programmatore, di solito raggruppate a formare un set di strumenti specifici per l'espletamento di un determinato compito all'interno di un certo programma. La finalità è ottenere un'astrazione, di solito tra l'hardware e il programmatore o tra software a basso e quello ad alto livello semplificando così il lavoro di programmazione}
}



\newglossaryentry{helpdesk}
{
    name=\glslink{helpdesk}{Helpdesk},
    text=helpdesk,
    sort=helpdesk,
    description={Servizio clienti telematico}
}

\newglossaryentry{tfs}
{
    name=\glslink{tfs}{TFS},
    text=TFS,
    sort=Tfs,
    description={Team foundation server}
}

\newglossaryentry{tfvc}
{
    name=\glslink{tfvc}{TFVC},
    text=TFVC,
    sort=Tfvc,
    description={Team foundation version control}
}
\newglossaryentry{deploy}
{
    name=\glslink{deploy}{Deploy},
    text=deploy,
    sort=deploy,
    description={ consegna o rilascio al cliente, con relativa installazione e messa in funzione o esercizio, di una applicazione o di un sistema software tipicamente all'interno di un sistema informatico aziendale}
}

\newglossaryentry{Angular}
{
    name=\glslink{Angular}{Angular},
    text=Angular,
    sort=Angular,
    description={ Angular 2+ è una piattaforma open source per lo sviluppo di applicazioni web con licenza MIT, evoluzione di AngularJS. Sviluppato principalmente da Google, la sua prima release è avvenuta il 14 settembre 2016}
}

\newglossaryentry{Websocket}
{
    name=\glslink{Websocket}{Websocket},
    text=Websocket,
    sort=Websocket,
    description={tecnologia web che fornisce canali di comunicazione full-duplex attraverso una singola connessione TCP}
}
\newglossaryentry{Cloud}
{
    name=\glslink{Cloud}{Cloud},
    text=Cloud,
    sort=Cloud,
    description={paradigma di erogazione di servizi offerti on demand da un fornitore ad un cliente finale attraverso la rete Internet}
}

\newglossaryentry{BigData}
{
    name=\glslink{BigData}{BigData},
    text=BigData,
    sort=BigData,
    description={raccolta estesa di dati in termini di volume, velocità e varietà da richiedere tecnologie e metodi analitici specifici per l'estrazione di valore o conoscenza}
}

\newglossaryentry{Framework}
{
    name=\glslink{Framework}{Framework},
    text=Framework,
    sort=Framework,
    description={Architettura logica di supporto su cui un software può essere progettato e realizzato}
}

\newglossaryentry{Broker}
{
    name=\glslink{Broker}{Broker},
    text=Broker,
    sort=Broker,
    description={Sistema che distribuisce vari aspetti del software su nodi differenti tramite l'utilizzo di oggetti remoti}
}

\newglossaryentry{Stage-IT}
{
    name=\glslink{Stage-IT}{Stage-IT},
    text=Stage-IT,
    sort=Stage-IT,
    description={Evento organizzate presso Padova Fiere in collaborazione con varie aziende locali che cercano di far approcciare gli studenti al mondo lavorativo tramite attività di stage}
}
\newglossaryentry{Javadoc}
{
    name=\glslink{Javadoc}{Javadoc},
    text=Javadoc,
    sort=Javadoc,
    description={Javadoc è un applicativo incluso all'interno del Java Development Kit della Sun Microsystems, utilizzato per la generazione automatica della documentazione del codice sorgente scritto in linguaggio Java}
}
\newglossaryentry{stub}
{
    name=\glslink{stub}{Stub},
    text=stub,
    sort=stub,
    description={è una porzione di codice utilizzata in sostituzione di altre funzionalità software in quanto può simulare il comportamento di codice esistente o l'interfaccia COM, e temporaneo sostituto di codice ancora da sviluppare}
}
\newglossaryentry{callback}
{
    name=\glslink{callback}{Callback},
    text=callback,
    sort=callback,
    description={funzione o blocco di codice che viene passato come parametro ad un'altra funzione}
}
\newglossaryentry{Disco}
{
    name=\glslink{Disco}{Disco},
    text=Disco,
    sort=Disco,
    description={Tool di process mining sviluppato da Fluxicon}
}

\newglossaryentry{JSON}
{
    name=\glslink{JSON}{JSON},
    text=JSON,
    sort=JSON,
    description={JavaScript Object Notation. Formato adatto all'interscambio di applicazione client/server}
}

\newglossaryentry{unit test}
{
    name=\glslink{unit test}{Unit test},
    text=unit test,
    sort=unit test,
    description={l'attività di testing di singole unità software.}
}
\newglossaryentry{mockup}
{
    name=\glslink{mockup}{Mockup},
    text=mockup,
    sort=mockup,
    description={realizzazione a scopo illustrativo di un oggetto o sistema senza le complete funzioni originali}
}
\newglossaryentry{open-closed}
{
    name=\glslink{open-closed}{Open-closed},
    text=open-closed,
    sort=open-closed,
    description={principio cardine di S.O.L.I.D \textit{principle} attraverso il quale un sistema deve essere aperto alle estensioni ma chiuso alle modifiche.}
} % database di termini
\makeglossaries


%**************************************************************
% Impostazioni di graphicx
%**************************************************************
\graphicspath{{immagini/}} % cartella dove sono riposte le immagini


%**************************************************************
% Impostazioni di hyperref
%**************************************************************
\hypersetup{
    %hyperfootnotes=false,
    %pdfpagelabels,
    %draft,	% = elimina tutti i link (utile per stampe in bianco e nero)
    colorlinks=true,
    linktocpage=true,
    pdfstartpage=1,
    pdfstartview=FitV,
    % decommenta la riga seguente per avere link in nero (per esempio per la stampa in bianco e nero)
    %colorlinks=false, linktocpage=false, pdfborder={0 0 0}, pdfstartpage=1, pdfstartview=FitV,
    breaklinks=true,
    pdfpagemode=UseNone,
    pageanchor=true,
    pdfpagemode=UseOutlines,
    plainpages=false,
    bookmarksnumbered,
    bookmarksopen=true,
    bookmarksopenlevel=1,
    hypertexnames=true,
    pdfhighlight=/O,
    %nesting=true,
    %frenchlinks,
    urlcolor=webbrown,
    linkcolor=RoyalBlue,
    citecolor=webgreen,
    %pagecolor=RoyalBlue,
    %urlcolor=Black, linkcolor=Black, citecolor=Black, %pagecolor=Black,
    pdftitle={\myTitle},
    pdfauthor={\textcopyright\ \myName, \myUni, \myFaculty},
    pdfsubject={},
    pdfkeywords={},
    pdfcreator={pdfLaTeX},
    pdfproducer={LaTeX}
}

%**************************************************************
% Impostazioni di itemize
%**************************************************************
\renewcommand{\labelitemi}{$\ast$}

%\renewcommand{\labelitemi}{$\bullet$}
%\renewcommand{\labelitemii}{$\cdot$}
%\renewcommand{\labelitemiii}{$\diamond$}
%\renewcommand{\labelitemiv}{$\ast$}


%**************************************************************
% Impostazioni di listings
%**************************************************************
\lstset{
    language=[LaTeX]Tex,%C++,
    keywordstyle=\color{RoyalBlue}, %\bfseries,
    basicstyle=\small\ttfamily,
    %identifierstyle=\color{NavyBlue},
    commentstyle=\color{Green}\ttfamily,
    stringstyle=\rmfamily,
    numbers=none, %left,%
    numberstyle=\scriptsize, %\tiny
    stepnumber=5,
    numbersep=8pt,
    showstringspaces=false,
    breaklines=true,
    frameround=ftff,
    frame=single
} 


%**************************************************************
% Impostazioni di xcolor
%**************************************************************
\definecolor{webgreen}{rgb}{0,.5,0}
\definecolor{webbrown}{rgb}{.6,0,0}


%**************************************************************
% Altro
%**************************************************************

\newcommand{\omissis}{[\dots\negthinspace]} % produce [...]

% eccezioni all'algoritmo di sillabazione
\hyphenation
{
    ma-cro-istru-zio-ne
    gi-ral-din
}

\newcommand{\sectionname}{sezione}
\addto\captionsitalian{\renewcommand{\figurename}{Figura}
                       \renewcommand{\tablename}{Tabella}}

\newcommand{\glsfirstoccur}{\ap{{[g]}}}

\newcommand{\intro}[1]{\emph{\textsf{#1}}}

%**************************************************************
% Environment per ``rischi''
%**************************************************************
\newcounter{riskcounter}                % define a counter
\setcounter{riskcounter}{0}             % set the counter to some initial value

%%%% Parameters
% #1: Title
\newenvironment{risk}[1]{
    \refstepcounter{riskcounter}        % increment counter
    \par \noindent                      % start new paragraph
    \textbf{\arabic{riskcounter}. #1}   % display the title before the 
                                        % content of the environment is displayed 
}{
    \par\medskip
}

\newcommand{\riskname}{Rischio}

\newcommand{\riskdescription}[1]{\textbf{\\Descrizione:} #1.}

\newcommand{\risksolution}[1]{\textbf{\\Soluzione:} #1.}

%**************************************************************
% Environment per ``use case''
%**************************************************************
\newcounter{usecasecounter}             % define a counter
\setcounter{usecasecounter}{0}          % set the counter to some initial value

%%%% Parameters
% #1: ID
% #2: Nome
\newenvironment{usecase}[2]{
    \renewcommand{\theusecasecounter}{\usecasename #1}  % this is where the display of 
                                                        % the counter is overwritten/modified
    \refstepcounter{usecasecounter}             % increment counter
    \vspace{10pt}
    \par \noindent                              % start new paragraph
    {\large \textbf{\usecasename #1: #2}}       % display the title before the 
                                                % content of the environment is displayed 
    \medskip
}{
    \medskip
}

\newcommand{\usecasename}{UC}

\newcommand{\usecaseactors}[1]{\textbf{\\Attori Principali:} #1. \vspace{4pt}}
\newcommand{\usecasepre}[1]{\textbf{\\Precondizioni:} #1. \vspace{4pt}}
\newcommand{\usecasedesc}[1]{\textbf{\\Descrizione:} #1. \vspace{4pt}}
\newcommand{\usecasepost}[1]{\textbf{\\Postcondizioni:} #1. \vspace{4pt}}
\newcommand{\usecasealt}[1]{\textbf{\\Scenario Alternativo:} #1. \vspace{4pt}}

%**************************************************************
% Environment per ``namespace description''
%**************************************************************

\newenvironment{namespacedesc}{
    \vspace{10pt}
    \par \noindent                              % start new paragraph
    \begin{description} 
}{
    \end{description}
    \medskip
}

\newcommand{\classdesc}[2]{\item[\textbf{#1:}] #2}