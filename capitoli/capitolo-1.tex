% !TEX encoding = UTF-8
% !TEX TS-program = pdflatex
% !TEX root = ../tesi.tex

%**************************************************************
\chapter{Il contesto aziendale}
\label{cap:il-contesto-aziendale}
\section{L'azienda}
Siav è una delle più importanti realtà italiane di sviluppo software e di servizi informatici specializzata nella dematerializzazione e nella gestione documentale e nei processi digitali. Dal 1989 l'azienda sviluppa soluzioni che organizzano in modo efficace i contenuti, le informazione e i processi per aiutare diverse realtà nel loro miglioramento a livello gestionale. Presenta diverse filiali nel suolo italiano che cooperano tra loro per un fine comune. Siav inoltre punta molto sulla collaborazione tra aziende, in campo nazionale sia a livello pubblico che privato, ma anche a livello internazionale. Una fra tutti Microsoft.
I loro servizi puntano in primo luogo, ad una miglior gestione, controllo ed automazione di tutti i principali processi aziendali; in secondo luogo, ad offrire diverse soluzioni concrete anche a singoli privati che necessitano di sistemi di gestione ed integrazione per lo loro piccola realtà.
\begin{figure}[!h] 
	\centering 
	\includegraphics[width=1.1\columnwidth]{siav} 
	\caption{I prodotti offerti da Siav (\url{https://tinyurl.com/sm45xtu})}
\end{figure}
Siav si colloca all'interno del mercato come una tra la più importanti realtà italiane a livello di \textit{Enterprise Content Management} offrendo servizi per poter migliorare processi e gestioni aziendali spaziando da un gestore di caselle PEC, ad un applicativo di process mining denominato \textit{Bipod}, fino ad arrivare al loro prodotto di punta: Archiflow/Silloge: un software in continuo sviluppo, mantenuto e aggiornato da piccoli team che lavorano in sinergia per raggiungere un obiettivo comune. Archiflow/Silloge offre una soluzione alla gestione di una cospicua mole di documenti, categorizzandoli in varie sezioni permettendone un facile reperimento.
\section{Clientela rivolta a Siav}
La maggior parte dei clienti che si affidano a \textit{Siav} sono aziende che presentano il bisogno di automatizzarsi e migliorare i propri processi interni andando cosi ad incrementare la propria efficenza sotto l'aspetto lavorativo. Tali aziende sono di vario genere con diverse necessità: dal semplice ristorante per la gestione delle fattura elettroniche, ad una nota catena di supermercati per la gestione di cassetti fiscali, fino ad aziende metalmeccaniche per garatire una maggiore efficenza di propri processi interni. Per ogni settore, l'azienda è in continua ricerca di nuove opportunità e soluzioni che possano potare a miglioramenti aziendali, cercando di spaziare verso più ambienti lavorativi, allargando sempre più il proprio campo applicativo per portare verso di sè una clientela più varia. Questo ha portato l'azienda a dover standardizzare i propri prodotti in modo da poter coinvolgere una maggior porzione di clientela, per ogni settore.
Siav è catalogata come \textit{Software house} e presenta un ampio catalogo di prodotti atti a soddisfare le principali necessità organizzative e gestionali di un'azienda, sta poi al potenziale cliente valutarne l'acquisto in base alle proprie necessità.
\section {Processi aziendali}
In questa sezione sono descritti i pricipali processi e strumenti che ho visto utilizzare all'interno dell'azieda durante il mio periodo di permanenza, dividendoli in metodologie e strumenti.
\subsection{Metodologie di sviluppo}
L'azienda normalmente si ritrova a dover far fronte ad alcune problematiche derivate da bug, criticità o variazioni importanti di requisiti che possono essere riscontrate durante i vari processi aziendali ed incidere in modo significativo sull'andamento del processo di sviluppo.
Per cercare di venire incontro a ciò l'azienda ha adottato una metodologia di tipo \textit{Agile}, cercando di mantenere un atteggiamento flessibile rispetto ai processi e gli obiettivi fissati. Risulta quindi fondamentale l'organizzazione di incontri durante vari momenti della giornata o settimana in modo da confrontarsi sull'andamento delle attività e lo stato di avanzamento del prodotto. È consuetudine dei team di lavoro un confronto giornaliero tramite \textit{daily meeting}, per discutere su quanto fatto durante la giornata precedente indicando le eventuali criticità riscontrate, pianificando cosi gli obiettivi per la giornata odierna. Ad inizio settimana invece viene organizzato un \textit{weekly meeting} per constatare gli stati di avanzamento rispetto alla settimana precedente, per poi fissare gli obiettivi per la settimana successiva. Gli obiettivi, le attività preposte e le scadenza prefissate a livello complessivo di team vengono solitamente appesi in una lavagna posta in luogo ben visibile all'interno del reparto di sviluppo, per ordine di priorità; in questo modo è quindi possibile tener sott'occhio le principali attività. Il rapporto con il cliente è di vitale importanza per \textit{Siav}, essa infatti offre un ottimo servizio di assistenza clienti tramite il quale è possibile effettuare segnalazioni di \textit{bug} o assistenza sui proprio prodotti. 
\begin{figure}[!h] 
	\centering 
	\includegraphics[width=0.4\columnwidth]{agile} 
	\caption{Ciclo di vita della metodologia Agile (\url{https://tinyurl.com/tkqyaww})}
\end{figure}
\subsection{Strumenti di supporto }
L'azienda mette a disposizione diversi strumenti di supporto, allo scopo di gestire e tenere traccia nel migliore dei modi tutte le attività di progetto.
Facendo fede alla metodologia \textit{Agile}\footcite{Agile: https://it.wikipedia.org/wiki/Metodologia_agile}, la maggior parte degli strumenti servono a gestire tutti gli aspetti che riguardano il codice. Viene utilizzato \gls{tfs}\footcite{TFS: https://en.wikipedia.org/wiki/Azure_DevOps_Server}, uno strumento di \textit{Microsoft} per la gestione delle principali attivtà di progetto che offre un set di strumenti per la collaborazione. Tra le principali funzionalità offerte sono presenti: gestione del repository tramite tecnologie Git o \gls{tfvc}\footcite{TFVC: https://docs.microsoft.com/it-it/visualstudio/mac/tf-version-control?view=vsmac-2019}, gestione dei requisiti e strumenti di \textit{build} automatizzati. Tale strumento aderisce in maniera completa alle tecniche \textit{agile}, andando quindi ad integrarsi in modo solido all'interno della realtà lavorativa. Per quanto riguarda il tracciamente delle attività viene utilizzato \textit{Evernote}\footcite{Evernote: https://evernote.com/}: un software che permette la scrittura e la condivisione di note all'interno di un gruppo di utenti. Cosi facendo ogni membro del team di sviluppo ha sotto controllo ogni attività svolta dagli altri membri. Un ulteriore strumento utilizzato per il tracciamento dello attività è \textit{Google Docs}. Tramite quest'ultimo è possibile assegnare attività a tutti i membri che possiedono i privilegi per accedere al documento in questione. Solitamente i task assegnati presentano una struttura semplice: la data di creazione, la possibilità o meno di menzionare direttamente l'interessato a cui risulta assegnato il task ed un casella di risposta per poter descrivere la sua effettiva terminazione, oppure un messaggio di testo di altro genere, che solitamente indica le problematiche per cui il task non è stato possibile risolverlo. \textit{Google Docs}\footcite{Google Docs: https://docs.google.com/} non è l'unico strumento di \textit{ticketing} che utilizza l'azienda ma è stato quello che ho potuto visionare durante il priodo di stage.\\
Per poter garantire una cooperazione efficente tra tutto il personale viene spesso utilizzato \textit{TeamViewer}: un software per il controllo e la gestire altre macchine da remoto, in questo modo è possibile raggiungere un buon grado di cooperazione tra il personale dell'azienda, pur trovandosi in filiali diverse. Tale metodologia viene spesso utilizzata anche all'interno dell'area di \textit{testing} per fornire assistenza immediata ai clienti.
\subsection{Ambienti di sviluppo}
Per poter mantenere una buona qualità dei processi di sviluppo all'interno dell'azienda è di fondamentale importanza la suddivisione dei reparti. Ogni reparto ha la sua mansione specifica per ogni singolo prodotto offerto da \textit{Siav}. Oltre a questo tipo di suddivisione, è presente anche una distinzione per ruolo all'interno di ogni specifico prodotto, andando a distinguere le aree sviluppo \textit{front-end} e quella \textit{back-end} da quelle di \textit{testing}, con il supporto costante dei sistemisti che permettono il corretto svolgimento di ogni singolo processo.
\section {Tecnologie utilizzate}
Le tecnologie utilizzate dell'azienda per la realizzazione dei propri prodotti sono fondate principalmente sulla possibilità di un utilizzo tramite \textit{\gls{Cloud}}, dedicandosi principalmente sullo sviluppo di \textit{Webapp}; dovendo fornire servizi di tipo gestionali e organizzativi l'azienda si trova spesso a dover far fronte alla gestione dei cosiddetti \textit{\gls{BigData}}: fattore crucuiale per \textit{Siav} che interessa la maggior parte dei propri prodotti. Per far fronte a tali problemi l'azienda utilizzia tecnologie innovative e all'avanguardia. Quest'ultime possono essere raggruppate in due macrocategorie: \textit{Front-end} e \textit{Back-end}.
\subsection{\textit{Front-end}}
Da quel che ho potuto constatare durante l'attività di stage, per quanto riguarda lo sviluppo web lato \textit{client} viene utilizzata prevalentamente la piattaforma \textit{\gls{Angular}\footcite{Angular: https://angular.io/}}.
Quest'ultima è ideale per lo sviluppo di \textit{webapp} multipiattaforma, offrendo diverse funzionalità allo sviluppatore, potendosi adattare a svariate tipologie di architetture.
 Tramite il consistente numero di componenti prefabbricati presenti in rete e una vasta gamma di funzionalità e servizi disponibili, è possibile sviluppare un'interfaccia \textit{front-end} solida e di qualità, soddisfando nel miglior modo possibile le aspettative del cliente. Un'altro aspetto fondmentale di tale \textit{\gls{Framework}} è la possibilità di essere utilizzato su più dispositivi, non restringendo quindi il campo applicativo di \textit{Siav} al solo \textit{Desktop o Laptop} ma spaziando verso tutti principali dispositivi presenti sul mercato.
\subsection{\textit{Back-end}}
Per quanto riguarda lato back-end vengono utilizzate molteplici tecnologie in base alle necessità che prevede il software di sviluppo;  \textit{RabbitMQ}\footcite{RabbitMQ: https://www.rabbitmq.com/}: questa tecnologia è stata pensata principalmente per un'architettura a microservizi e viene utilizzata in più prodotti all'interno dell'azienda. \textit{RabbitMQ} viene definito come un \textit{\gls{Broker}} di messaggisticia: ossia un sistema che monitora la trasmissioni di messaggi tra servizi attraverso una coda in cui vengono memorizzate i messaggi prima di essere inviati.\\ Un'altro sistema utilizzato, restando in ambito di microservizi è \textit{Kubernetes}\footcite{Kubernetes: https://kubernetes.io/}: un sistema per la gestione di \textit{container} che viene utilizzato dall'azienda per effettuare la duplicazione di contenitori in modo da far fronte ad una cospicua mole di operazioni e richieste da parte di utenti. Per poter sfruttare al massimo l'ambiente \textit{Cloud}, su cui \textit{Siav} investe molte risorse, viene utilizzato \textit{Apache Kafka}\footcite{Apache Kafka: https://kafka.apache.org/}: tale sistema è in grando di gestire un gran numero di operazioni in tempo reali provienienti da diversi \textit{Client}, sia per quanto rigaurda la fase di lettura che scrittura. L'azienda quindi cura molto le tecnologie che utilizza all'interno del propri prodotti cercando di trarne il massimo vantaggio da ognuna di esse. 

\begin{figure}[!h] 
	\centering 
	\includegraphics[width=0.8\columnwidth]{rabbitmq} 
	\caption{Diagramm illustrativo sul funzionamento di RabbitMQ
	(\url{https://www.ionos.it/digitalguide/siti-web/programmazione-del-sito-web/rabbitmq/})}
\end{figure}
\section{Propensione all'innovazione}
\textit{Siav} è un'azienda che fa dell'innovazione un suo punto cardine. I prodotti che offre sono costantemente aggiornati, cercando di adattarsi alle necessità del cliente. Per poter far ciò l'azienda ha messo a disposizione un serivizio di \textit{\gls{helpdesk}} in modo da fornire assistenza in merito ai propri prodotti per i propri clienti. L'azienda inoltre è alla continua ricerca di nuove tecnologie per poterle integrare all'interno dei propri applicativi; è presente all'interno dell'organico un team di ricerca e sviluppo che analizza le varie piattaforme e tecnologie presenti sul mercato. Spesso e volentieri tali attività vengono contestualizzate all'interno di proposte di stage, in modo da poter osservare nel concreto se gli studi e le ricerche fatte in precedenza possano portare a benefici per i prodotti dell'azienda, facendo anche un'attenta analisi dei costi. \textit{Siav} offre inoltre la possibiltà ad altre aziende di partecipare ad eventi esplicativi per illustrare le proprie strategie future. Uno su tutti è il \textit{tech meeting}: un evento dedicato al confronto sulle principali novità legate alle tematiche \textit{Coud, DevOps, Automation, OpenSource} ed alla condivisione dei progetti di successo realizzati grazie alle soluzioni adottate da \textit{Siav}. Questo \textit{format} innovativo, ha visto quest'anno la prtecipazione di \textit{Google}, portando le loro competenza e la loro visione sui temi tecnologici più innovativi. Durante questi incontri viene dedicato un approfondimento particolare per quanto riguarda le attività di ricerca e sviluppo, presentando tutte le attività ed i progetti su cui l'azienda sta investendo le proprie risorse. Tutti questi fattori fanno intendere la propensione di \textit{Siav} verso l'innovazione e la crescita, cercando di rendere partecipe anche altre realtà aziendali.  