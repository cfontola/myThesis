% !TEX encoding = UTF-8
% !TEX TS-program = pdflatex
% !TEX root = ../tesi.tex

%**************************************************************
% Sommario
%**************************************************************
\cleardoublepage
\phantomsection
\pdfbookmark{Sommario}{Sommario}
\begingroup
\let\clearpage\relax
\let\cleardoublepage\relax
\let\cleardoublepage\relax

\chapter*{Sommario}

Il presente documento descrive il lavoro svolto durante il periodo di stage, dalla durata di circa trecentoquaranta ore, del laureando Carlo Fontolan presso l'azienda \textit{Siav S.p.A.}
Lo studente vuole descrivere con riflessioni critiche e oggettive quanto appreso e maturato dal punto di vista professionale e delle competenze in tutto l'arco di durata dello stage.
In primo luogo era richiesto lo sviluppo di una libreria di filtraggio in abito process mining, che verrà poi inserità all'interno di una architettura a microservizi.
In secondo luogo, è stata richiesta le riscrittura di alcune sezioni di un loro vecchio prodotto, in cui la libreria sopradescritta farà da protagonista, per quanto riguarda la sezione di filtraggio. Le caratteristiche richieste dall’azienda proponente erano che il prodotto fosse idoneo a girare su \textit{cloud} e che fosse costituito da un’architettura a microservizi e sviluppato con \textit{framework} all'avanguardia.
Siccome in corso d'opera sono sorte alcune criticità in merito allo sviluppo della libreria, in accordo con il tutor aziendale, abbiamo deciso di rinegoziare alcuni requisiti andando ad eliminare la sezione riguardante il  \textit{back-end}, facendo spazio ad uno \textit{stub} in modo da poter osservare il comportamento dell'interfaccia, soddisfando così gli obiettivi sia a livello aziendale che personale. 
\section*{Organizzazione del testo}
Qui di seguito sono elencate le principali tematiche che andrò ad affrontare all'interno della relazione finale suddivise in quattro capitoli:
\begin{description}
	\item[{\hyperref[cap:il-contesto-aziendale]{Il primo capitolo}}] illustra il contesto organizzativo e produttivo dell'azienda in cui ho effettuato lo stage. 
	\item[{\hyperref[cap:stage interno della strategia aziendale]{Il secondo capitolo}}] va ad iullustrare le tematiche che \textit{Siav} ha inteso affrontare tramite l'attività che mi è stata proposta, in relazione alla strategia aziendale.
	\item[{\hyperref[cap:Resoconto dello stage]{Il terzo capitolo}}] andrà a spiegare le principali attività svolte durante tutto l'arco dello stage.
	\item[{\hyperref[cap:valutazione-retrospettiva]{Il quarto capitolo}}] andrà ad effettuare una valutazione complessiva sull'attività svolta mettendola in relazione con l'ambiente universitario.
\end{description}
\newpage
\section*{Convenzioni tipografiche}
\begin{itemize}
	\item{I termini in lingua diversa dall'italiano sono stati posti in corsivo per segnalarne il proprio uso intenzionale.}
	\item { I termini che possono presentare un significato non banale saranno riportati alla fine del documento nella sezione del glossario.}
	\item {Ogni immagine o tabella presente all'interno di tale relazione che non sarà stata prodotta direttamente dal sottoscritto, verrà integrata con la propria fonte.}
	\item {I titoli o le intestazioni dei paragrafi iniziano con la lettera maiuscola.}
	\item {Per poter garantire una corretta paginazione, le pagine di intestazione dei capitoli avranno numero di pagina dispari.}
	
\end{itemize}

%\vfill
%
%\selectlanguage{english}
%\pdfbookmark{Abstract}{Abstract}
%\chapter*{Abstract}
%
%\selectlanguage{italian}

\endgroup			

\vfill

